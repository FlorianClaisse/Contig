\documentclass[a4paper,12pt]{article}
\usepackage[frenchb]{babel}
\usepackage[utf8]{inputenc}
\usepackage[T1]{fontenc} 
\usepackage{lmodern}
\usepackage{mathptmx}

\usepackage{amsmath}
\usepackage{etoolbox}
\usepackage{float}
\usepackage{geometry}
\usepackage{hyperref}
\hypersetup{
    colorlinks=true,
    linkcolor=blue,
    filecolor=magenta,      
    urlcolor=cyan,
    pdftitle={Overleaf Example},
    pdfpagemode=FullScreen,
    }
\usepackage{graphicx}
%\usepackage[disable]{todonotes}
\usepackage{todonotes}
\usepackage{titlesec}
\titleformat*{\section}{\Large\bfseries\sffamily}
\titleformat*{\subsection}{\large\bfseries\sffamily}
\titleformat*{\subsubsection}{\itshape\subsubsectionfont}

\geometry{margin=2cm}

\newcounter{besoin}

% Descriptif des besoins:
% 1 - Label du besoin pour référencement 
% 2 - Titre du besoin
% 3 - Description
% 4 - Gestion d'erreurs
% 5 - Spécifications tests
\newcommand{\besoin}[5]{%
  \refstepcounter{besoin}%
  \fbox{\parbox{0.95\linewidth}{%
    \begin{center}\label{besoin:#1}\textbf{\sffamily Besoin~\thebesoin~: #2}\end{center}
    \ifstrempty{#3}{}{\textbf{Description~:} #3\par}
    \vspace{0.5em}
    \ifstrempty{#4}{}{\textbf{Gestion d'erreurs~:} #4\par}
    \vspace{0.5em}
    \ifstrempty{#5}{}{\textbf{Tests~:} #5\par}
  }}
}

\newcommand{\refBesoin}[1]{%
  Besoin~\ref{besoin:#1}%
}

\title{\sffamily \textbf{Projet de Développement Logiciel}}
\author{L3 Informatique -- Université de Bordeaux}
\date{}

\begin{document}

\maketitle

\section{Introduction}

Ce document présente les besoins nécessaires au développement de plusieurs
programme pour du traitement de données dans le domaine de la bio-informatique.

\begin{description}
    \item[Programme Principal:] Le programme principal permet de sélectionner des sous programme grace au premier argument de la ligne de commande.
    
    \item[Sous programme:] Les sous programmes sont des programmes qui permettent de traiter des données. Chaque sous programme possède une ligne de commande qui lui ai prore qui permet de le configuer.
    
     Cette ligne de commande est à passer dans la ligne de commande du programme principal.
\end{description}

\end{document}
